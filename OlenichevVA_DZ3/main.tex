\documentclass{article}
\usepackage[letterpaper,top=2cm,bottom=2cm,left=3cm,right=3cm,marginparwidth=1.75cm]{geometry}
\usepackage[russian]{babel}
\usepackage{amsmath}
\usepackage{graphicx}
\usepackage[utf8]{inputenc}
\usepackage[colorlinks=true, allcolors=blue]{hyperref}
\usepackage[pdf]{graphviz}

\usepackage{ucs} 
\usepackage[T1]{fontenc}



\usepackage{listings}
\usepackage{color}
\usepackage{minted}

\definecolor{dkgreen}{rgb}{0,0.8,0}
\definecolor{gray}{rgb}{0.5,0.5,0.5}
\definecolor{mauve}{rgb}{0.58,0,0.82}


\usepackage{mathtools}
\DeclarePairedDelimiter\ket{\lvert}{\rangle}



\usepackage{graphicx}



\title{ТМВ ДЗ №3}
\author{А-13б-19 Оленичев Владимир}
\date{}


\begin{document}
\maketitle





\section*{1. Машины Тьюринга}

\subsection*{1.1 Операции с языками и символами}

Реализуйте машины Тьюринга, которые позволяют выполнять следующие операции:
\begin{enumerate}

    
    \item \textbf{Сложение двух унарных чисел (1 балл)}
    
        Алгоритм:
        \begin{enumerate}
            Движемся вправо, пока не встретили '+'.
            Заменяем '+' на 1 и движемся в конец.
            Находим конечную единицу и удаляем её.
            Движемся к голове.
        \end{enumerate}
        
        
        \begin{minted}[]{yaml}
            input: 1111+11 
            blank: ' '
            start state: start

            table:
    
            start:  
                1: R
                '+': {write: '1', R: go_tail}
            go_tail:
                1: R
                ' ': {L: del}
            del:
                 1: {write: ' ', L: go_head}
          
            go_head:
                  1: L
                  ' ': {R: fin}
             fin:
        \end{minted}
        



    \item \textbf{Умножение унарных чисел (1 балл)}
    
        Копируем за знак '=' единицы первого множителя столько раз, сколько единиц во втором.
        
        Алгоритм:
        \begin{enumerate}
           Движемся в конец выражения, ставим знак '=' и возвращаемся в начало. Переносим второй множитель за знак равенства столько раз, сколько единиц в первом. После каждого переноса восстанавливаем замененный на промежуточные символы второй множитель. Промежуточный символ 'х'. В конце удаляем левую часть выражения, в т.ч. знак '='.
           
        \end{enumerate}

        \begin{minted}[]{yaml}
  
  input: '11*111'
  blank: ' '
  start state: set_eq
  table:

    set_eq:
          [1,'*']: R
          ' ': {write: '=', L: go_head}
    
    go_head:
            [1,'*']: L
            [' ', x]: {R: go_mul}
    
    go_mul:
          1: {write: x, R: go_second}
          x: R
          '*': {L: go_first}
    
    go_second:
          1: R
          '*': {R: second}
    
    second:
          1: {write: x, R: to_ans}
          x: R
          =: {L: res_second}
    
    to_ans:
          [1, =]: R
          ' ': {write: 1, L: back_second}
    
    back_second:
          [1, =]: L
          x: {R: second}
          
    res_second:
          x: {write: 1, L}
          '*': {L: go_head}
    
    go_first:
          x: L
          ' ': {R: clear}
    
    clear:
          [1,'*',x]: {write: ' ', R}
          =: {write: ' ', R: fin}
    fin:
        \end{minted}

    
\end{enumerate}


\subsection*{1.2 Операции с языками и символами}
Реализуйте машины Тьюринга, которые позволяют выполнять следующие операции:

\begin{enumerate}

    
    \item Принадлежность к языку $L = \{ 0^n1^n2^n \}, n \ge 0$ (0.5 балла)
    
        Алгоритм:
        \begin{enumerate}
            Все первые вхождения 0, 1, 2 заменяем на 'x'. Возвращаемся в начало.
            Повторяем предыдущий шаг, пока слово не будет заменено на все 'x' (иначе слово не принадлежит языку).
             1: слово принадлежит языку 0: нет.
             Пустое слово тоже принадлежит языку. 
        \end{enumerate}
        
        \begin{minted}[]{yaml}
input: '001122'
blank: ' '
start state: start


table:
  start:
    ' ': {L: good_word}    
    0: {write: X, R: go_one} 
    [1, 2]: {R: bad_word}
    X: R    
  
  go_one:
    1: {write: X, R: go_two}
    [' ', 2]: {R: bad_word}
    [X, 0]: R
  
  go_two:
    2: {write: X, L: go_head}
    ' ': {R: bad_word}
    [X, 1]: R
  
  go_head:
    ' ': {R: start}
    [0,1,2,X]: L
  
  good_word:
    ' ': {write: 1, L: to_fin}
    X: {write: ' ', L}
  
  bad_word:
    ' ': {L: clear}
    [0, 1, 2, X]: R
  
  clear:
    ' ': {write: 0, L: to_fin}
    [0, 1, 2, X]: {write: ' ', L} 

  to_fin:
    ' ': {R: fin}
  
  fin:
        \end{minted}

        

   
    \item Проверка соблюдения правильности скобок в строке (минимум 3 вида скобок) (0.5 балла)
    
        Алгоритм:
        \begin{enumerate}
            
            Ищем первую закрывающую скобку. Заменяем её на 'x'. Возвращаемся в начало.
            Ищем открывающую скобку такого же вида. Заменяем её на 'x'. 
            1: слово принадлежит языку (все 'x'), 0: нет
            Пустое слово - правильная скобочная последовательность.
        \end{enumerate}
        
        \begin{minted}[]{yaml}
input: '(([{}])'
blank: ' '
start state: start


table:
  start:
    ' ': {L: ok}    
    ['(', '[', '{']: {R: search_rb}
    [')', ']', '}']: {L: bad}
    
  search_rb:
    ' ': {L: eol}    
    ['(', '[', '{', 'x']: R
    ')': {write: 'x', L: rb_1}
    ']': {write: 'x', L: rb_2}
    '}': {write: 'x', L: rb_3}
  
  rb_1:
    ' ': {R: bad}
    '(': {write: 'x', R:  search_rb}
    ['[', '{']: {L: bad}
    'x': L
  
  rb_2:
    ' ': {R: bad}
    '[': {write: 'x', R: search_rb}
    ['(', '{']: {L: bad}
    'x': L

  rb_3:
    ' ': {R: bad}
    '{': {write: 'x', R: search_rb}
    ['[', '(']: {L: bad}
    'x': L 
    
  eol:
    ['(', '[', '{']: {L: bad} 
    'x': L
    ' ': {R: ok}
    
  bad:
    ['(', ')', '[', ']', '{', '}', 'x']: {write: ' ', R}
    ' ': {R: go_head}
  
  go_head:
    ['(', ')', '[', ']', '{', '}', 'x']: {write: ' ', R: go_head}
    ' ': {write: 0, L: fin}
    
  ok:
    ' ': {write: 1, L: fin}
    'x': {write: ' ', R}

  fin:

        \end{minted}


   
    \item Поиск минимального по длине слова в строке (слова состоят из символов 1 и 0 и разделены пробелом) (1 балл)
    
        Алгоритм:
        \begin{enumerate}
            Обрабатываем первые 2 слова, сравниваем их длину. Если первое слово больше, стираем его и восстанавливаем второе слово из промежуточных 'a' и 'b'. Если второе больше, то копируем на его место первое, после удаляем остатки. Повторяем до конца строки.
        \end{enumerate}
        
        \begin{minted}[]{yaml}
input: '1010 010 1'
blank: ' '
start state: first_word
table:
  
  first_word:
    0: {write: a, R: to_second}
    1: {write: b, R: to_second}
    [a, b]: R
    ' ': {L: first_is_small}
  
  to_second:
    [0, 1]: R
    ' ': {R: second_word}
 
  second_word:
    ' ': {L: one_left}
    0: {write: a, L: to_first}
    1: {write: b, L: to_first}
    [a, b]: {R: second_not_null}
  
  second_not_null:
    [a, b]: R
    0: {write: a, L: to_first}
    1: {write: b, L: to_first}
    ' ': {L: second_is_small}
  
  to_first:
    [a, b]: L
    ' ': {L: to_begin_first}

  to_begin_first:
    [0, 1, a, b]: L
    ' ': {R: first_word}

  one_left:
    ' ': {L: restore_and_exit}

  restore_and_exit:
    a: {write: 0, L}
    b: {write: 1, L}
    [0, 1]: L
    ' ': {R: fin}
  
  # первое меньше, замена 2-го на 1-е
  first_is_small:
    [a, b]: L
    ' ': {R: restore_first}

  restore_first:
    a: {write: 0, R}
    b: {write: 1, R}
    ' ': {R: cut_second}

  cut_second:
    [a, b, 0, 1]: {write: a, R}
    ' ': {L: return_and_copy}

  return_and_copy:
    a: L
    ' ': {L: copy_first}

  copy_first:
    [a, b]: L
    0: {write: a, R: carry0}
    1: {write: b, R: carry1}
    ' ': {R: delete_to_word}

  carry0:
    [a, b]: R
    ' ': {R: carry0_in_second}

  carry0_in_second:
    a: R
    [0, 1, ' ']: {L: set0_and_return}

  set0_and_return:
    a: {write: 0, L: return_and_copy}
    ' ': {L: return_and_copy}

  carry1:
    [a, b]: R
    ' ': {R: carry1_in_second}
  carry1_in_second:
    a: R
    [0, 1, ' ']: {L: set1_and_return}
  set1_and_return:
    a: {write: 1, L: return_and_copy}
    ' ': {L: return_and_copy}

  delete_to_word:
    [a, b]: {write: ' ', R}
    [0, 1]: {L: to_begin_first}
    ' ': {R: delete_to_word_in_sec}

  delete_to_word_in_sec:
    [a, b]: {write: ' ', R}
    [0, 1]: {L: to_begin_first}
    ' ': {R: fin}
  
  # 2-е меньше
  second_is_small:
    [a, b]: L
    ' ': {L: to_begin_first_and_del}

  to_begin_first_and_del:
    [0, 1, a, b]: L
    ' ': {R: delete_first}

  delete_first:
    [0, 1, a, b]: {write: ' ', R}
    ' ': {R: restore_second}

  restore_second:
    a: {write: 0, R}
    b: {write: 1, R}
    ' ': {L: to_begin_first}
  fin:

        \end{minted}

    
\end{enumerate}


\section*{2 Квантовые вычисления}
\subsection*{2.1 Генерация суперпозиций 1 (1 балл)}
    Дано $N$ кубитов ($1 \le N \le 8$) в нулевом состоянии $\ket{0\dots0}$. 
    Также дана некоторая последовательность битов, которое задаёт ненулевое базисное состояние размера $N$. Задача получить суперпозицию нулевого состояния и заданного.
    
    $$\ket{S} = \frac{1}{\sqrt2}(\ket{0\dots0} +\ket{\psi})$$
    
    То есть, требуется реализовать операцию, которая принимает на вход:
    \begin{enumerate}
        \item Массив кубитов $q_s$
        \item Массив битов $bits$ описывающих некоторое состояние $\ket{\psi}$. Это массив имеет тот же самый размер, что и $q_s$. Первый элемент этого массива равен $1$.
    \end{enumerate}
    
    Решение:
        
    \begin{enumerate}
         В начале у нас есть $N$ незаисимых кубитов $\ket{0}$
         Первые кубиты векторов различны, применим оператор Адамара к первому кубиту
         Все кубиты $qs$ равны 0 $\Rightarrow$ если кубит $bits[i] = 1$, то нужно запутать $i$-ый кубит, а если кубит $bits[i] = 0$, то не нужно, т.к кубиты совпадают и равны 0.

    \end{enumerate}
    
    \textbf{Прога}
    
    \begin{lstlisting}
        namespace Solution {
            open Microsoft.Quantum.Primitive;
            open Microsoft.Quantum.Canon;
            operation Solve (qs : Qubit[], bits : Bool[]) : Unit 
            {
                body
                {
                    H(qs[0]);
                    for i in 1..Length(qs) - 1 {
                        if (bits[i]) {
                            CNOT(qs[0], qs[i]);
                        }
                    }
                }
            }
        }
    \end{lstlisting}
    


\newpage


\subsection*{2.2 Различение состояний 1 (1 балл)}

    Дано $N$ кубитов ($1 \le N \le 8$), которые могут быть в одном из двух состояний:
    
    $$\ket{GHZ} = \frac{1}{\sqrt2}(\ket{0\dots0} +\ket{1\dots1})$$
    $$\ket{W} = \frac{1}{\sqrt N}(\ket{10\dots00}+\ket{01\dots00} + \dots +\ket{00\dots01})$$
    
    Требуется выполнить необходимые преобразования, чтобы точно различить эти два состояния. Возвращать $0$, если первое состояние и 1, если второе. 
    
    Решение:
        
    \begin{enumerate}
         Чтобы измерить состояние системы надо измерить кубиты
         При $N$ > 1 cостояние 1: $N$ нулей, либо $N$ единиц, cостояние 2: 1 единица
        При $N$ = 1 состояния не различить (в обоих состояниях может выпасть вектор, который содержит одну единицу)

    \end{enumerate}
    
    \textbf{Прога}

    \begin{lstlisting}
        namespace Solution {
            open Microsoft.Quantum.Primitive;
            open Microsoft.Quantum.Canon;
            operation Solve (qs : Qubit[]) : Int 
            {
                body
                {
                    mutable ones = 0;
                    for i in 0..Length(qs) - 1 {
                        if (M(qs[i]) == One) {  // measurement
                            set ones += 1;
                        }
                    }
                    if (ones == 1) {
                        return 1;
                    }
                    return 0;
                }
            }
        }
    \end{lstlisting}
 
    



\end{document}